%-----------------------------------LICENSE------------------------------------%
%   This file is part of Mathematics-and-Physics.                              %
%                                                                              %
%   Mathematics-and-Physics is free software: you can redistribute it and/or   %
%   modify it it under the terms of the GNU General Public License as          %
%   published by the Free Software Foundation, either version 3 of the         %
%   License, or (at your option) any later version.                            %
%                                                                              %
%   Mathematics-and-Physics is distributed in the hope that it will be useful, %
%   but WITHOUT ANY WARRANTY; without even the implied warranty of             %
%   MERCHANTABILITY or FITNESS FOR A PARTICULAR PURPOSE.  See the              %
%   GNU General Public License for more details.                               %
%                                                                              %
%   You should have received a copy of the GNU General Public License along    %
%   with Mathematics-and-Physics.  If not, see <https://www.gnu.org/licenses/>.%
%------------------------------------------------------------------------------%

% Use the standalone class for displaying the tikz image on a small PDF.
\documentclass[crop, tikz]{standalone}

% Needed for mathbb.
\usepackage{amssymb}

% Import the tikz package to use for the drawing.
\usepackage{tikz}

% Load the arrows.meta library.
\usetikzlibrary{arrows.meta}

% Begin the document.
\begin{document}

    % Draw the figure.
    \begin{tikzpicture}[>=LaTeX]

        % Draw the x-axis for the first figure..
        \draw[<-, thick] (-5.0, 0.0) to (-0.5, 0.0);
        \draw[->, thick] ( 0.5, 0.0) to ( 5.0, 0.0);

        % Shade an interval around the bottom origin.
        \draw[blue] (-0.5, 0.0) to[out=0,in=180] (0.0, -0.5)
                                to[out=0,in=180] (0.5,  0.0);

        % Connect the x-axis to the top origin.
        \draw (-0.5, 0.0) to[out=0,in=180] (0.0, 0.5)
                          to[out=0,in=180] (0.5, 0.0);

        % Shade the lower origin black, indicating it is not in the set.
        \draw[fill=black] (0.0,  0.5) circle (0.5mm);

        % Shade the upper origin blue, indicating is belongs to the set.
        \draw[fill=blue]  (0.0, -0.5) circle (0.5mm);

        % Label for the set U_0.
        \node at (2, 0.5) {$\mathcal{U}_{0}$};

        % Draw the second figure, shifted 2cm downwards.
        \begin{scope}[yshift=-2cm]
            % Draw the x-axis.
            \draw[<-, thick] (-5.0, 0) to (-0.5, 0);
            \draw[->, thick] ( 0.5, 0) to ( 5.0, 0);

            % Connect the lower origin to the x-axis.
            \draw (-0.5, 0.0) to[out=0,in=180] (0.0, -0.5)
                              to[out=0,in=180] (0.5,  0.0);

            % Shade an interval around the upper origin.
            \draw[blue] (-0.5, 0.0) to[out=0,in=180] (0.0, 0.5)
                                    to[out=0,in=180] (0.5, 0.0);

            % Color the two origins accordingly.
            \draw[fill=blue]  (0.0,  0.5) circle (0.5mm);
            \draw[fill=black] (0.0, -0.5) circle (0.5mm);

            % Label for the set U_1.
            \node at (2, 0.5) {$\mathcal{U}_{1}$};
        \end{scope}

        % Draw the third figure shifted 4cm downwards.
        \begin{scope}[yshift=-4cm]
            % Draw the x-axis.
            \draw[<-, thick] (-5.0, 0) to (-0.5, 0);
            \draw[->, thick] ( 0.5, 0) to ( 5.0, 0);

            % Blue coloring for the intersection of U_0 and U_1.
            \draw[blue, line width=2pt] (-0.5, 0) to (0.5, 0);

            % Delete the origin from this line.
            \draw[fill=white, draw=white] (0, 0) circle (0.5mm);

            % Color the two origins black.
            \draw[fill=black] (0,  0.5) circle (0.5mm);
            \draw[fill=black] (0, -0.5) circle (0.5mm);

            % Label for the intersection of U_0 and U_1.
            \node at (2, 0.5) {$\mathcal{U}_{0}\cap\mathcal{U}_{1}$};
        \end{scope}
    \end{tikzpicture}
\end{document}
