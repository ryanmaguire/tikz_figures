%-----------------------------------LICENSE------------------------------------%
%   This file is part of tikz_figures.                                         %
%                                                                              %
%   tikz_figures is free software: you can redistribute it and/or              %
%   modify it it under the terms of the GNU General Public License as          %
%   published by the Free Software Foundation, either version 3 of the         %
%   License, or (at your option) any later version.                            %
%                                                                              %
%   tikz_figures is distributed in the hope that it will be useful,            %
%   but WITHOUT ANY WARRANTY; without even the implied warranty of             %
%   MERCHANTABILITY or FITNESS FOR A PARTICULAR PURPOSE.  See the              %
%   GNU General Public License for more details.                               %
%                                                                              %
%   You should have received a copy of the GNU General Public License along    %
%   with tikz_figures.  If not, see <https://www.gnu.org/licenses/>.           %
%------------------------------------------------------------------------------%

% Use the standalone class for displaying the tikz image on a small PDF.
\documentclass[crop, tikz]{standalone}

% Import the tikz package to use for the drawing.
\usepackage{tikz}

% Tikz packages used.
\usetikzlibrary{decorations.markings, arrows.meta}

% Begin the document.
\begin{document}

    % Draw the figure.
    \begin{tikzpicture}[%
        ->-/.style = {%
            decoration = {%
                markings,
                mark = at position 0.55 with \arrow{Stealth}
            },
            postaction = {decorate}
        }
    ]

        % Nodes for the first cycles.
        \coordinate (0) at (0.000:2);
        \coordinate (2) at (72.00:2);
        \coordinate (6) at (144.0:2);
        \coordinate (3) at (216.0:2);
        \coordinate (4) at (288.0:2);

        % Dots to mark the elements of the cycle.
        \draw[fill = black] (0) circle (0.05);
        \draw[fill = black] (2) circle (0.05);
        \draw[fill = black] (6) circle (0.05);
        \draw[fill = black] (3) circle (0.05);
        \draw[fill = black] (4) circle (0.05);

        % Label everything.
        \node at (0.000:2.3) {$0$};
        \node at (72.00:2.3) {$2$};
        \node at (144.0:2.3) {$6$};
        \node at (216.0:2.3) {$3$};
        \node at (288.0:2.3) {$4$};

        % Draw the first cycle (its a pentagon).
        \draw[->-] (0) to (2);
        \draw[->-] (2) to (6);
        \draw[->-] (6) to (3);
        \draw[->-] (3) to (4);
        \draw[->-] (4) to (0);

        % The second cycle is a triangle. Shift to draw it.
        \begin{scope}[xshift = 6cm]

            % Nodes for the cycle.
            \coordinate (1) at (90.00:2);
            \coordinate (5) at (210.0:2);
            \coordinate (7) at (330.0:2);

            % Dots to mark the points.
            \draw[fill = black] (1) circle (0.05);
            \draw[fill = black] (5) circle (0.05);
            \draw[fill = black] (7) circle (0.05);

            % Labels.
            \node at (90.00:2.3) {$1$};
            \node at (210.0:2.3) {$5$};
            \node at (330.0:2.3) {$7$};

            % And lastly, draw the cycle.
            \draw[->-] (1) to (5);
            \draw[->-] (5) to (7);
            \draw[->-] (7) to (1);
        \end{scope}
    \end{tikzpicture}
\end{document}
