%-----------------------------------LICENSE------------------------------------%
%   This file is part of tikz_figures.                                         %
%                                                                              %
%   tikz_figures is free software: you can redistribute it and/or              %
%   modify it it under the terms of the GNU General Public License as          %
%   published by the Free Software Foundation, either version 3 of the         %
%   License, or (at your option) any later version.                            %
%                                                                              %
%   tikz_figures is distributed in the hope that it will be useful,            %
%   but WITHOUT ANY WARRANTY; without even the implied warranty of             %
%   MERCHANTABILITY or FITNESS FOR A PARTICULAR PURPOSE.  See the              %
%   GNU General Public License for more details.                               %
%                                                                              %
%   You should have received a copy of the GNU General Public License along    %
%   with tikz_figures.  If not, see <https://www.gnu.org/licenses/>.           %
%------------------------------------------------------------------------------%

% Use the standalone class for displaying the tikz image on a small PDF.
\documentclass[crop, tikz]{standalone}

% Import the tikz package to use for the drawing.
\usepackage{tikz}

% Tikz packages used.
\usetikzlibrary{decorations.markings, arrows.meta}

% Begin the document.
\begin{document}

    % Draw the figure.
    \begin{tikzpicture}[%
        ->-/.style = {%
            decoration = {%
                markings,
                mark = at position 0.55 with \arrow{Stealth}
            },
            postaction = {decorate}
        }
    ]

        % Coordinates for the points.
        \foreach\x in {0, 60, ..., 300}{%

            % The point on the circle, given in polar form using degrees.
            \coordinate (\x) at (\x:2);

            % Add dots for each point.
            \draw[fill = black] (\x) circle (0.05);

            % Label the points by their angle on the unit circle.
            \node at (\x:2.4) {$\x$};
        }

        % Dashed lines for acting by a.
        \draw[densely dashed] (0) to (60);
        \draw[densely dashed] (120) to (180);
        \draw[densely dashed] (240) to (300);

        % Solid lines for acting by b.
        \draw[->-] (60) to (120);
        \draw[->-] (180) to (240);
        \draw[->-] (300) to (0);
    \end{tikzpicture}
\end{document}
