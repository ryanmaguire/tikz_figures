%-----------------------------------LICENSE------------------------------------%
%   This file is part of tikz_figures.                                         %
%                                                                              %
%   tikz_figures is free software: you can redistribute it and/or              %
%   modify it it under the terms of the GNU General Public License as          %
%   published by the Free Software Foundation, either version 3 of the         %
%   License, or (at your option) any later version.                            %
%                                                                              %
%   tikz_figures is distributed in the hope that it will be useful,            %
%   but WITHOUT ANY WARRANTY; without even the implied warranty of             %
%   MERCHANTABILITY or FITNESS FOR A PARTICULAR PURPOSE.  See the              %
%   GNU General Public License for more details.                               %
%                                                                              %
%   You should have received a copy of the GNU General Public License along    %
%   with tikz_figures.  If not, see <https://www.gnu.org/licenses/>.           %
%------------------------------------------------------------------------------%

% Use the standalone class for displaying the tikz image on a small PDF.
\documentclass[crop, tikz]{standalone}

% Import the tikz package to use for the drawing.
\usepackage{tikz}

% The markings and arrows libraries will be used, so load those.
\usetikzlibrary{decorations.markings, arrows.meta}

% Begin the document.
\begin{document}

    % Draw the figure.
    \begin{tikzpicture}[%
        > = latex,%
        font = \footnotesize,%
        line width = 0.3pt,%
        line cap = round,%
        ->-/.style = {% This style is for a line with an arrow in the center.
            decoration = {%
                markings,%
                mark = at position 0.55 with \arrow{Stealth}%
            },%
            postaction = {decorate}%
        }%
    ]

        % Draw the x and y axes.
        \draw[->] (-0.1in, 0in) to (2.2in, 0.0in);
        \draw[->] (0.0in, -0.1in) to (0.0in, 2.1in);

        % Mark the axes with a few numbers.
        \draw (1.00in, 0.03in) to (1.00in, -0.03in) node[below] {1};
        \draw (2.00in, 0.03in) to (2.00in, -0.03in) node[below] {2};
        \draw (0.03in, 1.00in) to (-0.03in, 1.00in) node[left] {1};
        \draw (0.03in, 2.00in) to (-0.03in, 2.00in) node[left] {2};

        % Draw the curve y^2 = x.
        \draw[rotate = 90, draw = blue, ->-]
            (0.0in, 0.0in) parabola (1.4141in, -2.0in);

        % Draw perpendicular lines to indicate the position of the point.
        \draw[densely dashed] (2.0in, 0.0in) to (2.0in, 1.4141in);
        \draw[densely dashed] (0.0in, 1.4141in) to (2.0in, 1.4141in);

        % Label the function.
        \node at (1.0in, 0.8in) {$y^{2}=x$};
    \end{tikzpicture}
\end{document}
